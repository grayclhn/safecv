% In general, remove the sections and subsections that are empty.
% This is based on the CV advice at
% <http://theprofessorisin.com/2012/01/12/dr-karens-rules-of-the-academic-cv/>
\documentclass[12pt]{article}%
\author{Firstname McLastname}
\title{Curriculum Vitae}
\date{\today}

\frenchspacing
\usepackage{
    calc,
    hyperref,
    ragged2e,
    newtxtext, % <- default safe `professional` font.
    newtxmath,
    url,
}
\usepackage[T1]{fontenc}
\usepackage[margin=1in]{geometry}
\usepackage[letterspace=30]{microtype}

\RaggedRight
\urlstyle{same} %<- Do not monospace urls.
\newcommand{\allcaps}[1]{\textls[30]{\MakeUppercase{#1}}}

% Set spacing and size of sections and subsections. (Don't use
% anything below subsection.)
\usepackage[tiny,raggedright]{titlesec}
\titleformat{\section}{\bf\lsstyle\uppercase}{}{}{}{}
\titlespacing*{\section}{0pt}{2\baselineskip}{\baselineskip}
\titlespacing*{\subsection}{0pt}{\baselineskip}{\itemsep}

% Set up list formatting. Renew `\descfont` to change handling of the
% labels in description lists. If you change fonts to one that uses
% proportional number spacing, you'll need to set it to tabular in
% these labels.
\usepackage[shortlabels,inline]{enumitem}
\newlength\dgap
\newlength\dwidth
\setlength{\dgap}{\widthof{ -- }}
\setlength{\dwidth}{\widthof{2000}}
\setenumerate{topsep=0pt,parsep=0pt,partopsep=0pt,leftmargin=0pt}
\setitemize{topsep=0pt,parsep=0pt,partopsep=0pt,leftmargin=0pt,label={}}
\setdescription{font=\normalfont,topsep=0pt,parsep=0pt,partopsep=0pt,leftmargin=!,labelwidth=\dwidth,labelsep=\dgap}
% Use `\p` to insert periods that don't take up space, e.g. `2009\p`.
\newcommand{\p}{\rlap{.}}

% Change the displayed title to emphasize the author and radically
% reduce spacing.
\makeatletter
\def\maketitle{%
\begin{center}%
\par{\textls{\MakeUppercase{\textbf{\large\@author}}}}%
\vspace{\itemsep}%
\par{\@title, \@date}%
\end{center}%
}
\makeatother

\usepackage{tabularx}

\begin{document}
\maketitle
\section*{Contact Information}
\begin{tabular}{@{}lrl@{}}
Building and office number & Office  & 123-456-7890                         \\
Your Department            & Mobile  & 123-456-790                          \\
Your University            & Email   & \url{youraccount@youruniversity.edu} \\
City, State Zipcode        & Website & \url{http://example.edu/~youraccount} %<- If you don't have one yet, get a simple web page.
\end{tabular}

\subsection*{Department Contacts}
\begin{tabular}{@{}llll@{}}
Placement Director & Director's Name & \url{account@university.edu} & 123-456-790 \\
Graduate Coordinator & Coordinator's Name & \url{account@university.edu} & 123-456-790
\end{tabular}

\section*{Education}

\begin{description}[noitemsep]
\item[year] PhD Your Field. Your University (expected)
\item Dissertation title: \textit{Several essays on my research} %<- If you don't have a title, make one up
\item Advisor: Advisor name
\item[year] \allcaps{MS} Your Field. Your University (If appropriate)
\item[year] \allcaps{BA} Undergraduate Major. Undergraduate University
\end{description}

\section*{Research and Teaching Interests}
% You should be prepared to teach a graduate elective in the subject
% listed as 'Interest 1' and to have some enthusiasm about teaching
% undergraduate courses in the other subjects listed here.
Interest 1; Interest 2; Interest 3

\section*{Honors and Awards}

\begin{description}[noitemsep]
\item[year] Name of most recent award
\item[year] Name of second most recent award, etc.
% Keep going in reverse chronological order. Do not include awards
% from undergrad unless they were *extremely* impressive
\end{description}

% \section*{Grants and Fellowships}
% \begin{description}[noitemsep]
% \item[year] Details of most recent grant or fellowship
% \item[year] Details of second most recent grant or fellowship
% % Keep going in reverse chronological order
% \end{description}

\section*{Publications} %<- you can also just call this 'research'
% Uncomment any that are appropriate, then add bibliographical information
%
% \subsection*{Books}
% \subsection*{Edited Volumes}
% \subsection*{Refereed Journal Articles}
% \begin{itemize}
% \item Bibliographic details for article 1
% \item Bibliographic details for article 2
% \end{itemize}
% \subsection*{Book Chapters}
% \subsection*{Conference Proceedings}
% \subsection*{Encyclopedia Entries}
% \subsection*{Book Reviews}
% \subsection*{Manuscripts in Submission}
% \begin{itemize}
% \item Article 1. Revise and resubmit at \textit{Journal}
% \item Article 2. Submitted to \textit{Journal}
% \end{itemize}
\subsection*{Job Market Paper}

``Name of job market paper'' (joint with coauthors, if any)

\vspace{\itemsep}

\smallcaps{Abstract}: People often put the abstract of their job market
paper here. It is optional. But don't do anything weird like putting
it all in italics or small caps.
  
\subsection*{Other Manuscripts in Preparation}
% You should only list something here if it is actually on your
% website or could be emailed to a stranger on short notice.
\begin{itemize}
\item Name of other paper 1
\item Name of other paper 2, etc.
\end{itemize}

% \subsection*{Popular writings}
% \begin{itemize}
% \item Op-Ed 1, with bibliographical information
% \item Op-Ed 2, with bibliographical information
% \end{itemize}

\section*{External Presentations}
% List in reverse chronological, as always; uncomment the relevant
% categories, if any.
%
% \subsection*{Invited Seminars}
% \begin{description}
% \item[year] List the name of each seminar for that year, with (month/day) in parentheses
% \item[year] List the name of each seminar for that year (which is
% older than the previous line), with (month/day) in parentheses
% \end{description}
% \subsection*{Invited Conference Presentations}
% % List in reverse chronological, as always
% \begin{description}
% \item[year] Same rules for listing as for the ``Invited Seminars''
% \end{description}
% \subsection*{Other Presentations}
%% If this is the only category, remove the ``Other presentations''
%% subsection heading and just list this directly under ``External
%% presentations''
\begin{description}
\item[year] List the name of each seminar for that year, with
(month/day) in parentheses. This list should not include presentations
you have given at your university, only off-campus presentations.
\end{description}

\section*{Professional Experience}
% Do not list course number, just course title. You can add
% clarifications like "Masters level" or "MBA" for non-undergraduate
% teaching. e.g.
%
% \item MBA Econometrics 1 (2 times)
%
% even if it is not the official title of the class. (As long as it's
% accurate!)
%
% \subsection*{Courses Taught as Primary Instructor}
% \begin{itemize}[noitemsep]
% \item Course title (number of times you have taught it)
% \item Other course title (n)
% \end{itemize}
%
% If you've been a TA for several graduate classes, you may want to
% put that separately.
\subsection*{Teaching Assistant}
Class 1, Class 2, Class 3, etc.

\subsection*{Research Assistant}
% List, as always, in reverse chronological order. Especially relevant
% internships can go here, consulting jobs, RA work, etc.
\begin{description}
\item[start\,--\,end] Research Assistant for Prof 1 and Prof 2. You can
add a single sentence explaining what you did, if it was interesting.
\item[start\,--\,end] Another job. You can add another sentence
explaining what you did.
\end{description}

\section*{Professional Service}
% \subsection*{Editorial positions}
% \begin{description}[noitemsep]
% \item[start\,--\,end] Position, \textit{Journal name}
% \end{description}

\subsection*{Journal Referee}
% Put the most prestigous journals first! Don't bury Econometrica in
% the middle of the list.
\textit{Journal title 1};
\textit{Journal title 2}

% \subsection*{University Service}
% % Reverse chronological order
% \begin{description}[noitemsep]
% \item[start\,--\,end] Position (no details.)
% \end{description}
% \subsection*{Departmental Service}
% % Reverse chronological order
% \begin{description}[noitemsep]
% \item[start\,--\,end] Position (no details.)
% \end{description}

\subsection*{Professional Membership}
List the professional societies you belong to.

% \section*{Media Coverage}
% % Reverse chronological order
% \begin{itemize}
% \item Bibliographic information for the article, if you were written
% up in the \emph{New York Times}. (Or on the news, or whatever.)
% \item Bibliographic information for the second article.
% \end{itemize}
\section*{Languages}
% Optional, obviously, and leave it out if the only one is "English"
\begin{description}[noitemsep]
\item[Language:] Fluency, etc.
\end{description}

\section*{References}

\begin{itemize*}
\item
\begin{tabular}{@{}l}
  Name 1                  \\
  Address 1               \\
  123-456-7890            \\
  \url{email@example.edu}
\end{tabular}
\item 
\begin{tabular}{l}
  Name 2                  \\
  Address 2               \\
  123-456-7890            \\
  \url{email@example.edu}
\end{tabular}
\item 
\begin{tabular}{l}
  Name 3                  \\
  Address 3               \\
  123-456-7890            \\
  \url{email@example.edu}
\end{tabular}
\end{itemize*}
\end{document}
